\documentclass[a4paper,12pt]{article}
\usepackage[utf8]{inputenc}
\usepackage[brazil]{babel}
\usepackage[lmargin=3cm,tmargin=3cm,rmargin=2cm,bmargin=2cm]{geometry} 
\usepackage[T1]{fontenc} 
\usepackage{amsmath,amsthm,amsfonts,amssymb,dsfont,mathtools,blindtext} %pacotes matemáticos
\usepackage{blindtext}
\usepackage{graphicx}

\begin{document}
\title{\textbf{Lista 2 - Fundamentos de Matemática para Computação} \\\vspace{5}\small UFG - Instituto de Informática\\\large Professor Dr. Julliano Rosa Nascimento}
\author{Alessandra dos Santos Morais, Filipe Brito Manganelli,\\Iury Alexandre Alves Bó, Nelsi de Sousa Barbosa Junior}
\date{9 de agosto de 2021}
\maketitle


\begin{itemize} %função para organizar as questões em itens
    \item [] \textbf{Questão 1 -} Encontre as fórmulas ou regra simples para construir uma sequência de inteiros que comece com a lista dada. Assumindo que sua resposta esteja correta, determine os três próximos termos da sequência. %para cada questão criar um item desse modelo com o enunciado da questão
    \
     \\\\\textbf{a)} 1, 2, 2, 3, 3, 3, 4, 4, 4, 4, . . . %criar um texto dessa forma para cada item das questões
     \\\\Resolução:
     \\\\Podemos representar a fórmula dessa sequência como $A_n=\lfloor\sqrt{2\cdot n}+1/2\rfloor$
     \\em que, $A_n$ = o termo da sequência; $n$ = posição do termo, começando com $n=1$.
     \\\\Próximos três termos:
     \\\\$A_1_1=\lfloor\sqrt{2\cdot 11}+0,5\rfloor$
     \\$A_1_1=\lfloor4,69+0,5\rfloor$
     \\$A_1_1=\lfloor5,19\rfloor$
     \\$A_1_1=5$ 
    \\\\$A_1_2=\lfloor\sqrt{2\cdot 12}+0,5\rfloor$
     \\$A_1_2=\lfloor4,89+0,5\rfloor$
     \\$A_1_2=\lfloor5,39\rfloor$
     \\$A_1_2=5$ 
    \\\\$A_1_3=\lfloor\sqrt{2\cdot 13}+0,5\rfloor$
     \\$A_1_3=\lfloor5,09+0,5\rfloor$
     \\$A_1_3=\lfloor5,59\rfloor$
     \\$A_1_3=5$ 
     
     \\\\
     \newpage
     \textbf{b)} 1, 10, 11, 100, 101, 110, 111, 1000, 1001, 1010, 1011, . . .
     \\\\Resolução:
     \\\\Podemos representar a fórmula dessa sequência binária como $A_n=n+1$
     \\sendo uma Progressão Aritmética(P.A) de razão = 1
     \\em que, $A_n$ = o termo da sequência (na base 2); $n$ = posição do termo, começando com $n=0$.
     \\\\Próximos três termos:
     \\\\$A_1_1=1011_2+1_2 = 1100_2$
     \\\\$A_1_2=1100_2+1_2 = 1101_2$
     \\\\$A_1_3=1101_2+1_2 = 1110_2$
     \\
     
     \\\\
     \\\\\textbf{c)} 1, 3, 15, 105, 945, 10395, 135135, 2027025, 34459425, . . .
     \\\\Resolução:
    \\\\Podemos representar a fórmula dessa sequência como $A_n=A_n_-_1\cdot (3+2\cdot (n-1))$
    \\em que, $A_n$ = o termo da sequência; $n$ = posição do termo, começando com $n=0$ e $A_0=1$.
     \\\\Próximos três termos:
     \\\\$A_9=A_8\cdot (3+2\cdot (9-1))$
     \\$A_9=34459425\cdot (3+2\cdot 8)$
     \\$A_9=34459425\cdot 19$
     \\$A_9=654729075$
     \\\\$A_1_0=A_9\cdot (3+2\cdot (10-1))$
     \\$A_1_0=654729075\cdot (3+2\cdot 9)$
     \\$A_1_0=654729075\cdot 21$
     \\$A_1_0=13749310575$
     \\\\$A_1_1=A_1_0\cdot (3+2\cdot (11-1))$
     \\$A_1_1=13749310575\cdot (3+2\cdot 10)$
     \\$A_1_1=13749310575\cdot 23$
     \\$A_1_1=316234143225$
     \\
     
     \\\\ 
     \item [] \textbf{Questão 2 -} Encontre uma fórmula fechada para:
     \\
     
     \begin{centering}
     
     $\dfrac{1}{1\cdot 2} + \dfrac{1}{2\cdot 3} + ... + \dfrac{1}{n\cdot (n + 1)}$
     
     \end{centering}
     
     examinando os valores dessa expressão para pequenos valores de n.
     \newline
    Para encontrar a resposta, você deve testar para alguns valores de n, por exemplo:
    
    \item Para $n = 1$, temos que $\frac{1}{1\cdot 2} = \frac{1}{2}$
    \item Para $n = 2$, temos que $\frac{1}{1\cdot 2} + \frac{1}{2\cdot 3} = \frac{1}{2} + \frac{1}{6} = \frac{4}{6} = \frac{2}{3}$
    \item Para $n = 3$, temos que $\frac{1}{1\cdot 2} + \frac{1}{2\cdot 3} + \frac{1}{3\cdot 4} = \frac{1}{2} + \frac{1}{6} + \frac{1}{12} = \frac{9}{12} = \frac{3}{4}$
    \item ...
    \\\\Resolução:
    \\\\Podemos representar esse somatório como $A_n=\sum\limits_{j=1}^n \frac{1}{j\cdot (j+1)}$
    \\Seguindo a lógica dada pelo exercício temos a sequência até o $6^o$ termo como
    \begin{center}
    $\left(\dfrac{1}{2},\dfrac{2}{3},\dfrac{3}{4},\dfrac{4}{5},\dfrac{5}{6},\dfrac{6}{7}\right)$
    \end{center}
    Observando o padrão da sequência podemos concluir a fórmula $A_n=\dfrac{n}{n+1}$
    \\começando com $n=1$.
    
\end{itemize}
\end{document}
