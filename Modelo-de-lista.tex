\documentclass[a4paper,12pt]{article}
\usepackage[utf8]{inputenc}
\usepackage[brazil]{babel}
\usepackage[lmargin=3cm,tmargin=3cm,rmargin=2cm,bmargin=2cm]{geometry} 
\usepackage[T1]{fontenc} 
\usepackage{amsmath,amsthm,amsfonts,amssymb,dsfont,mathtools,blindtext} %pacotes matemáticos
\usepackage{blindtext}
\usepackage{graphicx}

\begin{document}

\title{\textbf{Lista X} \\\vspace{5}\small UFG - Instituto de Informática\\\large Professor <NOME DOSCENTE>}

\author{Filipe Brito Manganelli, Manga Nelas}

\date{9 de agosto de 2021}

\maketitle

\begin{itemize} %função para organizar as questões em itens
    \item []\textbf{Questão 1 -} %item = NULL
     
     \\\\\textbf{a)} %criar um texto dessa forma para cada item das questões
     \\\\Resolução:
     \\\\ %para cada resposta pular \\\\(duas) linhas
    
    %resolução do exercício
     
     \\ %pular \\(uma) linha para passar para o próximo item 
     
     \item [] \textbf{Questão 2 -}
     \\\\Resolução:
     \\\\
     
     
     \\ %próxima questão
\end{itemize}
    
\end{document}
